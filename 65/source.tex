\documentclass[12pt]{article}

\usepackage{amsmath,amsthm,amssymb,mathrsfs}
\usepackage[hidelinks]{hyperref}
\usepackage[margin=1.1in]{geometry}

\newtheorem{theorem}{Theorem}
\newtheorem{assumption}[theorem]{Assumption}
\newtheorem{lemma}[theorem]{Lemma}
\newtheorem{corollary}[theorem]{Corollary}
\theoremstyle{definition}
\newtheorem{definition}[theorem]{Definition}
\theoremstyle{remark}
\newtheorem{remark}[theorem]{Remark}

\newcommand{\Cyc}{\mathcal{C}}
\newcommand{\E}{\mathrm{E}}
\newcommand{\V}{\mathrm{V}}

\title{A note on the harmonic sum of cycle lengths in graphs with $kn$ edges}
\author{---}
\date{4th of February 2026}

\begin{document}
\maketitle

\begin{abstract}
Let $G$ be a graph on $n$ vertices with $m:=|E(G)|=kn$ edges (so $k=m/n$), and let
$a_1<a_2<\cdots$ be the distinct lengths of cycles appearing in $G$.
Define the harmonic sum of cycle lengths by $L(G):=\sum_i 1/a_i$.
Assuming an exact extremal statement (quoted in a survey of Montgomery)
that for large $d$ the complete bipartite graph $K_{d,n-d}$ minimises
$L(G)$ among graphs with at least $d(n-d)$ edges, we give a short,
fully explicit proof that $L(G)\gg \log k$ (indeed $L(G)\ge \tfrac12\log k-O(1)$).
\end{abstract}

\section{Definitions and the Erd\H{o}s--Hajnal harmonic-sum parameter}

Throughout, graphs are finite and simple.

\begin{definition}[Cycle-length set and harmonic sum]
Let $G$ be a graph.
Define the \emph{cycle-length set}
\[
\Cyc(G):=\{\ell\in \mathbb{N}:\text{$G$ contains a cycle of length $\ell$}\}.
\]
Define the \emph{harmonic sum of cycle lengths}
\[
L(G):=\sum_{\ell\in \Cyc(G)}\frac{1}{\ell}.
\]
Equivalently, if $a_1<a_2<\cdots$ lists the distinct elements of $\Cyc(G)$,
then $L(G)=\sum_i 1/a_i$.
\end{definition}

This parameter was proposed by Erd\H{o}s and Hajnal as a measure of
how ``dense'' the set of cycle lengths is. A classical theorem of
Gy\'arf\'as--Koml\'os--Szemer\'edi (1984) shows that $L(G)\ge c\log \delta(G)$
for graphs of sufficiently large minimum degree $\delta(G)$, and yields
corresponding statements for density/average-degree assumptions.
See \cite{GKS84}.

In this note we do \emph{not} reprove the Gy\'arf\'as--Koml\'os--Szemer\'edi
argument. Instead, we show that once one assumes the (claimed) \emph{exact}
extremal result identifying the minimiser of $L(G)$ at a given edge threshold,
the lower bound $L(G)\gg \log k$ becomes an immediate computation.

\section{The extremal hypothesis}

The following statement is quoted (as ``forthcoming work'') in Montgomery's
survey on cycles and expansion \cite{MontgomerySurvey}. We take it as a
black-box assumption.

\begin{assumption}[Exact extremal minimiser at the $d(n-d)$ threshold]\label{ass:extremal}
There exists an integer $d_0$ and a function $n_0:\mathbb{N}\to\mathbb{N}$ such
that for every integer $d\ge d_0$ and every $n\ge n_0(d)$, every $n$-vertex graph
$H$ with
\[
|E(H)|\ge d(n-d)
\]
satisfies
\[
L(H)\ \ge\ L\bigl(K_{d,n-d}\bigr),
\]
with equality attained uniquely by $H\cong K_{d,n-d}$.
\end{assumption}

\begin{remark}
This is precisely the ``exact extremal result corresponding to Theorem 2.3''
described in \cite{MontgomerySurvey}, i.e.\ the minimiser is the complete
$bipartite$ graph with parts of sizes $d$ and $n-d$.
\end{remark}

\section{Cycle lengths in complete bipartite graphs}

We now determine $\Cyc(K_{d,n-d})$ and hence compute $L(K_{d,n-d})$.

\begin{lemma}[Cycle-length spectrum of $K_{d,n-d}$]\label{lem:cycle-spectrum}
Let $2\le d\le n-d$. Then
\[
\Cyc\bigl(K_{d,n-d}\bigr)=\{4,6,8,\dots,2d\}.
\]
\end{lemma}

\begin{proof}
First, $K_{d,n-d}$ is bipartite, so every cycle alternates between the
two parts and therefore has even length. Thus $\Cyc(K_{d,n-d})\subset 2\mathbb{N}$.

If a cycle has length $2t$, it uses exactly $t$ vertices in each part.
Since the smaller part has size $d$, necessarily $t\le d$, so $2t\le 2d$.
Hence no cycle length exceeds $2d$.

Conversely, fix any integer $t$ with $2\le t\le d$. Choose distinct vertices
$x_1,\dots,x_t$ in the $d$-vertex part and distinct vertices $y_1,\dots,y_t$
in the $(n-d)$-vertex part. Since the graph is complete bipartite, all edges
$x_i y_j$ exist, and in particular
\[
x_1y_1x_2y_2\cdots x_ty_tx_1
\]
is a cycle of length $2t$. This shows every even length $2t$ with $2\le t\le d$
occurs. Therefore $\Cyc(K_{d,n-d})=\{4,6,\dots,2d\}$.
\end{proof}

\begin{corollary}[Harmonic sum of cycle lengths in $K_{d,n-d}$]\label{cor:L-Kdn}
For $2\le d\le n-d$,
\[
L\bigl(K_{d,n-d}\bigr)
=\sum_{t=2}^{d}\frac{1}{2t}
=\frac12\left(H_d-1\right),
\]
where $H_d:=\sum_{t=1}^{d}\frac{1}{t}$ is the $d$th harmonic number.
\end{corollary}

\begin{proof}
By Lemma \ref{lem:cycle-spectrum},
\[
L\bigl(K_{d,n-d}\bigr)=\sum_{\ell\in\{4,6,\dots,2d\}}\frac{1}{\ell}
=\sum_{t=2}^{d}\frac{1}{2t}
=\frac12\sum_{t=2}^{d}\frac{1}{t}
=\frac12(H_d-1).
\]
\end{proof}

\section{A standard lower bound for harmonic numbers}

\begin{lemma}[Integral lower bound]\label{lem:harmonic-lb}
For every integer $d\ge 1$,
\[
H_d\ \ge\ \log(d+1).
\]
\end{lemma}

\begin{proof}
Since $x\mapsto 1/x$ is decreasing on $(0,\infty)$, for each integer $t\ge 1$
and all $x\in[t,t+1]$ we have $1/x\le 1/t$. Hence
\[
\int_t^{t+1}\frac{dx}{x}\ \le\ \int_t^{t+1}\frac{dx}{t}=\frac{1}{t}.
\]
Summing from $t=1$ to $d$ yields
\[
\int_1^{d+1}\frac{dx}{x}
=\sum_{t=1}^{d}\int_t^{t+1}\frac{dx}{x}
\le \sum_{t=1}^{d}\frac{1}{t}
=H_d.
\]
The integral equals $\log(d+1)$, giving $H_d\ge \log(d+1)$.
\end{proof}

\section{Main implication: $L(G)\gg \log k$ from the extremal minimiser}

We now prove the desired lower bound for graphs with $kn$ edges.

\begin{theorem}\label{thm:main}
Assume Assumption \ref{ass:extremal}. Let $G$ be a graph on $n$ vertices with
$m:=|E(G)|$ edges, and set $k:=m/n$.
Let $a_1<a_2<\cdots$ be the distinct cycle lengths in $G$.
Then there exists $k_0$ (depending only on $d_0$ from Assumption \ref{ass:extremal})
such that whenever $k\ge k_0$ and $n\ge n_0(\lfloor k\rfloor)$,
\[
\sum_i \frac{1}{a_i}=L(G)\ \ge\ \frac12\log k - O(1),
\]
and in particular $L(G)\gg \log k$ as $k\to\infty$.
\end{theorem}

\begin{proof}
If $k$ is bounded (say $1\le k\le 2$), then $\log k=O(1)$ and the statement
$L(G)\gg \log k$ is trivial after adjusting constants. Hence we assume $k\ge 2$.

Write $m:=|E(G)|$ and $k=m/n$. Since $m\le \binom{n}{2}$, we have
\[
k\le \frac{n-1}{2}.
\]
In particular $d:=\lfloor k\rfloor\le (n-1)/2< n/2$, so $d\le n-d$ and the
standing hypothesis of Lemma~\ref{lem:cycle-spectrum} is satisfied.

Let $d:=\lfloor k\rfloor$. Then $d\le k$ and $d\ge 1$.
We claim that
\[
|E(G)|\ge d(n-d).
\]
Indeed,
\[
|E(G)|=m=kn\ \ge\ dn,
\]
and since $n-d\le n$ we have $d(n-d)\le dn$. Thus $|E(G)|\ge d(n-d)$.

If furthermore $d\ge d_0$ and $n\ge n_0(d)$ (in particular, $n\ge n_0(\lfloor k\rfloor)$),
Assumption \ref{ass:extremal} implies
\[
L(G)\ \ge\ L\bigl(K_{d,n-d}\bigr).
\]
By Corollary \ref{cor:L-Kdn} and Lemma \ref{lem:harmonic-lb},
\[
L\bigl(K_{d,n-d}\bigr)=\frac12(H_d-1)\ \ge\ \frac12(\log(d+1)-1).
\]
Since $k\ge 2$ gives $\lfloor k\rfloor\ge k/2$, we have $d+1\ge k/2$ and so
\[
\log(d+1)\ \ge\ \log\Bigl(\frac{k}{2}\Bigr)=\log k-\log 2.
\]
Therefore, for $k\ge \max\{2,d_0\}$,
\[
L(G)\ \ge\ \frac12(\log(d+1)-1)
\ \ge\ \frac12(\log k-\log 2-1)
= \frac12\log k - \frac12(\log 2+1).
\]
This is of the form $\frac12\log k - O(1)$, and it implies $L(G)\gg \log k$.
\end{proof}

\begin{remark}[Sharpness up to the constant $1/2$]
For the extremal graph itself,
\[
L\bigl(K_{d,n-d}\bigr)=\sum_{t=2}^{d}\frac{1}{2t}
=\frac12(H_d-1)\sim \frac12\log d.
\]
Thus, even with the best possible argument, one cannot replace the coefficient
$1/2$ in front of $\log k$ by any larger absolute constant in general (up to
lower-order terms).
\end{remark}

\section*{Acknowledgement of context}
The original lower bound $L(G)\gg \log d$ (for average degree/minimum degree
parameters) was proved by Gy\'arf\'as--Koml\'os--Szemer\'edi in \cite{GKS84}.
Montgomery's survey \cite{MontgomerySurvey} discusses later improvements and
states the exact extremal characterisation (Assumption \ref{ass:extremal})
as forthcoming work; assuming that characterisation, the $\Omega(\log k)$
bound for the edge-density parameter follows directly from evaluating
$L(K_{d,n-d})$.

\begin{thebibliography}{9}

\bibitem{MontgomerySurvey}
R.~Montgomery,
\newblock \emph{Cycles and expansion in graphs},
\newblock EMS Magazine 138 (2025), DOI: 10.4171/MAG/287.

\bibitem{GKS84}
A.~Gy\'arf\'as, J.~Koml\'os, and E.~Szemer\'edi,
\newblock \emph{On the distribution of cycle lengths in graphs},
\newblock J.\ Graph Theory 8 (1984), 441--462.

\end{thebibliography}

\end{document}
