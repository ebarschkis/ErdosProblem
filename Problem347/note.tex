\documentclass[12pt]{article}
\usepackage{amsmath,amssymb,amsthm}
\usepackage{geometry}
\usepackage{hyperref}
\geometry{margin=1in}

\title{A Sequence with Doubling Ratio and Full-Density Subset Sums}
\author{Enrique Barschkis}
\date{January 21, 2026}

\newtheorem{lemma}{Lemma}

\begin{document}
\maketitle

\begin{abstract}
We construct a nondecreasing sequence of integers
$A=\{a_1\le a_2\le\cdots\}$ with
\[
\lim_{n\to\infty}\frac{a_{n+1}}{a_n}=2
\]
such that for every cofinite subsequence $A'\subseteq A$, the finite
subset-sum set
\[
P(A')=\left\{\sum_{x\in B}x: B\subseteq A' \text{ finite}\right\}
\]
has asymptotic density $1$ in $\mathbb{N}$.
\end{abstract}

\section{Problem}
Does there exist a nondecreasing sequence $A$ of integers with
\[
\lim_{n\to\infty}\frac{a_{n+1}}{a_n}=2
\]
such that $P(A')$ has density $1$ for every cofinite subsequence
$A'\subseteq A$?

\section{Construction by Blocks}

\subsection{Block lengths}
Define
\[
k_n := 4+\left\lceil \log_2\log_2(n+16)\right\rceil
\qquad (n\ge0).
\]
Then $k_n\to\infty$ slowly and
$2^{k_n}\asymp \log n$ as $n\to\infty$.

\subsection{Block scales}
Let $M_0:=10$ and define recursively
\[
M_{n+1}:=\left\lfloor \left(2^{k_n}-\tfrac32\right)M_n \right\rfloor.
\]
Since $M_n\ge10$, we have
\begin{equation}
(2^{k_n}-2)M_n<M_{n+1}<(2^{k_n}-1)M_n.
\tag{1}
\end{equation}

\subsection{Block entries}
Block $n$ consists of
\[
M_n,\ 2M_n,\ 4M_n,\ \dots,\ 2^{k_n-2}M_n,\ (2^{k_n-1}-1)M_n+1.
\]
Concatenating all blocks yields a nondecreasing sequence $A$.

\section{Limit of Consecutive Ratios}

Inside a block, consecutive ratios equal $2$ except at the final step:
\[
\frac{(2^{k_n-1}-1)M_n+1}{2^{k_n-2}M_n}
=2-\frac{1}{2^{k_n-2}}
+\frac{1}{2^{k_n-2}M_n}
=2+o(1).
\]
Across block boundaries,
\[
\frac{M_{n+1}}{(2^{k_n-1}-1)M_n+1}
=2+O(2^{-k_n})+O(M_n^{-1}).
\]
As $k_n, M_n \to \infty$, all ratios tend to $2$, hence
\[
\lim_{n\to\infty}\frac{a_{n+1}}{a_n}=2.
\]

\section{Density for Cofinite Subsequences}

Let $A'$ be cofinite in $A$. Then there exists $n_0$ such that all blocks
with $n\ge n_0$ are contained in $A'$. Set $T:=A_{\ge n_0}$.

\subsection{Digit sets}
Fix $n\ge n_0$ and write $k=k_n$. Define
\[
c_n := (2^{k-1}-1)M_n.
\]
Set
\[
B_n=\left\{jM_n \mid 0\le j\le 2^{k-1}-1\right\}
\cup
\left\{jM_n+1 \mid 2^{k-1}\le j\le 2^k-2\right\}.
\]
Then $|B_n|=2^k-1$, $c_n\in B_n$, $c_n+1\notin B_n$, and
$B_n\subset[0,M_{n+1})$.

\begin{lemma}[Greedy covering]
For every integer $x$ with $0\le x\le M_{n+1}$, there exists
$b\in B_n$ such that
\[
0\le x-b\le M_n.
\]
\end{lemma}

\subsection{Greedy expansion}

Fix $N\ge n_0$. For each $m\le M_{N+1}$ define recursively
\[
r_{N+1}:=m,
\qquad
b_n:=\max\{b\in B_n:b\le r_{n+1}\},
\qquad
r_n:=r_{n+1}-b_n,
\]
for $n=N,\dots,n_0$. Then
\begin{equation}
m=\sum_{n=n_0}^N b_n + d,
\qquad 0\le d\le M_{n_0}.
\tag{2}
\end{equation}

\subsection{Correction lemma}

\begin{lemma}
If at least $M_{n_0}$ indices satisfy $b_n=c_n$, then $m\in P(T)$.
\end{lemma}
\begin{proof}
Since each block contains both unshifted multiples $jM_n$ and shifted
ones $jM_n+1$, the choice $b_n=c_n$ gives access to a $+1$ adjustment.
With at least $M_{n_0}$ such indices, the remainder $d\le M_{n_0}$ can
be absorbed by switching the block sum $c_n$ to the last block element
$c_n+1$ in at most $d$ positions.
\end{proof}

\subsection{Counting exceptions}

Let $E_N$ denote the number of integers $m\le M_{N+1}$
not in $P(T)$. We give a detailed bound showing
\[
\frac{E_N}{M_{N+1}}\longrightarrow 0
\quad (N\to\infty).
\]

Fix $N\ge n_0$ and set
\[
s_n:=|B_n|=2^{k_n}-1,
\qquad
a_n:=s_n-1=2^{k_n}-2.
\]
For each $m\le M_{N+1}$ the greedy expansion produces a unique vector
\[
(b_{n_0},\dots,b_N)\in B_{n_0}\times\cdots\times B_N
\]
and a remainder $d\in[0,M_{n_0}]$, with
\[
m=\sum_{n=n_0}^N b_n+d.
\]
Thus the map $m\mapsto (b_{n_0},\dots,b_N,d)$ is injective. By the
correction lemma, $m\notin P(T)$ implies fewer than $M_{n_0}$ indices
with $b_n=c_n$. Therefore
\begin{equation}
E_N
\le
(M_{n_0}+1)\cdot
\#\{(b_{n_0},\dots,b_N):\#\{n:b_n=c_n\}<M_{n_0}\}.
\tag{3}
\end{equation}

For a fixed subset $I\subset\{n_0,\dots,N\}$ of size $j$, the number of
vectors with $b_n=c_n$ exactly for $n\in I$ is
\[
\prod_{n\notin I} a_n,
\]
since there are $a_n$ non-special choices at $n\notin I$. Summing over
all $I$ with $|I|=j$ and then over $j\le M_{n_0}-1$ gives
\begin{equation}
E_N
\le
(M_{n_0}+1)\left(\prod_{n=n_0}^N a_n\right)
\sum_{j=0}^{M_{n_0}-1} e_j\!\left(\frac{1}{a_{n_0}},\dots,\frac{1}{a_N}\right),
\tag{4}
\end{equation}
where $e_j$ is the $j$th elementary symmetric sum. Using the bound
\[
e_j(x_1,\dots,x_r)\le\frac{(x_1+\cdots+x_r)^j}{j!},
\]
we obtain
\[
E_N
\le
(M_{n_0}+1)\left(\prod_{n=n_0}^N a_n\right)
\sum_{j=0}^{M_{n_0}-1}\frac{S_N^j}{j!},
\qquad
S_N:=\sum_{n=n_0}^N\frac{1}{a_n}.
\]
On the other hand, the recurrence $M_{n+1}=\lfloor(a_n+\tfrac12)M_n\rfloor$ implies
$M_{n+1} \ge (a_n+\tfrac12)M_n - 1$. Writing
$M_{n+1}\ge (a_n+\tfrac12)M_n\left(1-\frac{1}{(a_n+\tfrac12)M_n}\right)$ and using
$k_n\ge 6$ (hence $a_n\ge 62$ and $M_{n+1}\ge a_n M_n$), we get
$M_n\ge 10\cdot 62^n$ and
$\sum_n\frac{1}{(a_n+\tfrac12)M_n}<\infty$. Therefore the product
$\prod_n\left(1-\frac{1}{(a_n+\tfrac12)M_n}\right)$ is bounded below by a
constant $c>0$, and hence
\[
M_{N+1}
\ge
c \cdot M_{n_0}\prod_{n=n_0}^N\left(a_n+\tfrac12\right).
\]
Hence
\begin{equation}
\label{eq:final_bound} % <--- Added label here
\frac{E_N}{M_{N+1}}
\le
\frac{C_{n_0}}{c}
\left(\prod_{n=n_0}^N\frac{a_n}{a_n+\tfrac12}\right)
\left(\sum_{j=0}^{M_{n_0}-1}\frac{S_N^j}{j!}\right).
\tag{5}
\end{equation}
where $C_{n_0} := (M_{n_0}+1)/M_{n_0}$. Since
\[
\log\left(\frac{a_n}{a_n+\tfrac12}\right)
=-\log\left(1+\frac{1}{2a_n}\right)
\le -\frac{1}{4a_n}
\]
for all $n$ (using $a_n\ge 2^{k_n}-2\ge 62$), we have
\[
\prod_{n=n_0}^N\frac{a_n}{a_n+\tfrac12}
\le
\exp\left(-\frac14 S_N\right).
\]
Because $2^{k_n}\asymp\log n$, we have $a_n\asymp\log n$ and therefore
\[
S_N=\sum_{n=n_0}^N\frac{1}{a_n}\asymp\sum_{n=n_0}^N\frac{1}{\log n}\to\infty.
\]
The exponential decay in \eqref{eq:final_bound} dominates the fixed polynomial % <--- Changed \label to \eqref
factor $\sum_{j=0}^{M_{n_0}-1}S_N^j/j!$, so
\[
\frac{E_N}{M_{N+1}}\to 0,
\]
and $P(T)$ has asymptotic density $1$.

\section{Conclusion}

The constructed sequence $A$ satisfies
\[
\lim_{n\to\infty}\frac{a_{n+1}}{a_n}=2,
\]
and for every cofinite subsequence $A'\subseteq A$, the set $P(A')$
has density $1$. Hence the answer to the original problem is
\emph{yes}.

\section*{Acknowledgments}
The author would like to thank Terence Tao and Wouter van Doorn, as the proof presented here builds upon their ideas.
\end{document}
